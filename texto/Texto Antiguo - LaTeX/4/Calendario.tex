% this file is called up by thesis.tex
% content in this file will be fed into the main document

\chapter{Calendario} % top level followed by section, subsection

% the code below specifies where the figures are stored
\ifpdf
    \graphicspath{{4/figures/PNG/}{4/figures/PDF/}{4/figures/}}
\else
    \graphicspath{{4/figures/EPS/}{4/figures/}}
\fi
% ----------------------- contents from here ------------------------

\section{Calendario de Actividades}

La programaci\'{o}n de actividades se refleja en el calendario o cronograma de actividad. Siendo el \'{u}nico integrante de las actividades el mismo autor de este proyecto. El \'{a}rea sombreada de las tablas determina la duraci\'{o}n estimada de la actividad, teniendo una X si fue realizada, o el campo vaci\'{o} si no se completo. Los campos que contienen un asterisco ( * ), son aquellas actividades programadas para  ser realizadas m\'{a}s adelante.
Destaca la revisi\'{o}n y correci\'{o}n del proyecto, que por atenci\'{o}n a otros pendientes escolares no se pudo realizar conforme lo acordado con el asesor. Esto afectando de gran manera al avance del trabajo.

\figuremacroW{1}{Calendario de Actividades realizadas y por terminar para Marzo-Junio}{}{}

En las actividades programadas para el verano solo han sido consideras las que requieren trabajo individual y no revisi\'{o}n por parte del asesor.

\figuremacroW{2}{Calendario de Actividades programadas para Julio-Septiembre}{}{}






% ---------------------------------------------------------------------------
% ----------------------- end of thesis sub-document ------------------------
% ---------------------------------------------------------------------------