% this file is called up by thesis.tex
% content in this file will be fed into the main document

\chapter{Estado del Arte} % top level followed by section, subsection


% ----------------------- contents from here ------------------------

Los peque\~{n}os sat\'{e}lites han capturado la imaginaci\'{o}n a nivel mundial de fabricantes y dise\~{n}adores de los mismos, debido a sus capacidades econ\'{o}micas y tecnol\'{o}gicas. En las universidades con intereses en tecnolog\'{i}a espacial el primer foco de atenci\'{o}n son los objetos espaciales considerados como micro, nano y pico sat\'{e}lites. Seg\'{u}n avances en la miniaturizaci\'{o}n electr\'{o}nica se permite tener cargas \'{u}tiles (payloads) e instrumentos para misiones espaciales de menor tama\~{n}o y masa, tambi\'{e}n de un consumo de energ\'{i}a m\'{a}s eficiente. Siendo popular el est\'{a}ndar CubeSat [1] (acr\'{o}nimo de Cube Satellite) dise\~{n}ado principalmente por Jordi Puig-Suari y Robert Twiggs, adem\'{a}s creador del concepto CanSat.


Recientemente en el pa\'{i}s se cuenta con la referencia escasa de proyectos relacionados con peque\~{n}os sat\'{e}lites. Siendo el primero, un trabajo sin publicaciones o avances por parte del Instituto de Rob\'{o}tica de Yucat\'{a}n con el desarrollo de un CubeSat (TRIY-SAT I), en segunda instancia se puede relacionar al proyecto SARSEM-ICARUS II o Sistema Aerost\'{a}tico de Repetici\'{o}n Sub-Espacial Mexicano que realiz\'{o} un viaje sub-orbital mediante un globo meteorol\'{o}gico, desarrollado por el Club de Radio Amateur del Estado de Guanajuato A.C. y liderado por los ingenieros Erick Arzola (XE1CKJ) y Jonathan Remba (XE1BRX); con participaci\'{o}n y apoyo de operadores de radio amateur dentro del pa\'{i}s y el extranjero. Durante 2008 se realizan dos proyectos por estudiantes del Tecnol\'{o}gico de Monterrey, campus Puebla y campus Ciudad de M\'{e}xico respectivamente; estas dos iniciativas crearon un veh\'{i}culo espacial de acuerdo a las normas de competencia CanSat. Teniendo lugar en la “Annual Cansat Competition” de 2008 en Texas, EUA.





% ---------------------------------------------------------------------------
% ----------------------- end of thesis sub-document ------------------------
% ---------------------------------------------------------------------------