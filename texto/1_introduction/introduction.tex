
% this file is called up by thesis.tex
% content in this file will be fed into the main document

%: ----------------------- introduction file header -----------------------
\chapter{Introducci\'{o}n}

% the code below specifies where the figures are stored
\ifpdf
    \graphicspath{{1_introduction/figures/PNG/}{1_introduction/figures/PDF/}{1_introduction/figures/}}
\else
    \graphicspath{{1_introduction/figures/EPS/}{1_introduction/figures/}}
\fi

% ----------------------------------------------------------------------
%: ----------------------- introduction content ----------------------- 
% ----------------------------------------------------------------------



%: ----------------------- HELP: latex document organisation
% the commands below help you to subdivide and organise your thesis
%    \chapter{}       = level 1, top level
%    \section{}       = level 2
%    \subsection{}    = level 3
%    \subsubsection{} = level 4
% note that everything after the percentage sign is hidden from output



\section{CanSat} % section headings are printed smaller than chapter names
% intro
CanSat es el acr\'{o}nimo de “Canned Satellite” (Sat\'{e}lite Enlatado). El concepto fue introducido por primera vez a finales de la d\'{e}cada de 1990 en la Universidad de Stanford [2]. La idea detr\'{a}s del proyecto fue permitir a estudiantes enfrentarse a algunas situaciones encontradas en construir un sat\'{e}lite, pero al mismo tiempo para ser resueltas en un periodo de tiempo menor y con un gasto m\'{i}nimo. Los estudiantes deb\'{i}an dise\~{n}ar y construir instrumentos, adaptarlos a una lata de bebidas para despu\'{e}s lanzarla en un cohete. Entonces la lata de aluminio junto a su instrumentaci\'{o}n caen al suelo atados de un paraca\'{i}das mientras realizan diferentes tipos de experimentos. Todo esto siendo posible en menos de un a\~{n}o y a un costo muy bajo.


Cada a\~{n}o se celebra una competencia CanSat en Estados Unidos de Am\'{e}rica (“Annual CanSat Competition”, Texas). Esta competencia permite a equipos de diferentes universidades y preparatorias tener los retos de dise\~{n}ar y construir un sistema aeroespacial, de acuerdo a las especificaciones dadas por el comit\'{e} organizador. As\'{i} tambi\'{e}n se celebra otra competencia de interacci\'{o}n internacional, llamada ARLISS \'{o} “A Rocket Launch For International Student Satellites”.
As\'{i} mismo, toman lugar otras competencias CanSat en distintos pa\'{i}ses de Europa y As\'{i}a. Siendo el continente europeo [12] donde un mayor \'{i}ndice de estudiantes y pa\'{i}ses son involucrados.



\section{Normas CanSat} % subsection headings are again smaller than section names
Se tendr\'{a} un mejor entendimiento de \'{e}ste proyecto al revisar con calma las normas definidas para un concurso de pico-sat\'{e}lites CanSat. De manera resumida, las normas explican el valor educativo de una competencia CanSat, descripci\'{o}n de la misi\'{o}n, agrupaci\'{o}n de estudiantes para los equipos, diferentes tipos de misi\'{o}n CanSat, requerimientos, clases de CanSat, evaluaci\'{o}n y fases del proyecto.


\subsection{Requerimientos Generales}
En base a las gu\'{i}as de diferentes concursos se puede determinar:
\begin{enumerate}
\item La instrumentaci\'{o}n debe ser construida para caber dentro de una lata de bebida de 12 oz. (355ml). 115mm de altura y 66 mm de diametro.
\item La masa m\'{a}xima est\'{a} limitada a 350 gramos.
\item El costo m\'{a}ximo es de 1000 USD.
\item El paraca\'{i}das debe ser colocado en la parte superior y las antenas en el fondo del CanSat.
\item Cada CanSat debe contar con su propia estaci\'{o}n en tierra para fines de comunicaci\'{o}n.
\item El CanSat debe ser apagado antes del vuelo y durante su colocaci\'{o}n en \'{o}rbita. Solo debe ser activado despu\'{e}s de su separaci\'{o}n de el cohete o globo. [12]
\item El CanSat debe resistir una aceleraci\'{o}n de hasta 20G
\item Los materiales explosivos o de f\'{a}cil combusti\'{o}n dentro del dispositivo est\'{a}n prohibidos.
\end{enumerate}

\subsection{Clasificaci\'{o}n de Sat\'{e}lites por su tama\~{n}o}
El termino peque\~{n}o sat\'{e}lite es usado para mencionar a todo aquel s\'{a}telite con una masa menor a 500kg. A\'{u}n as\'{i} una  clasificaci\'{o}n m\'{a}s detallada es la propuesta:

%\begin{table}[htdp]
%\centering
%\begin{tabular}{ccc} % ccc means 3 columns, all centered; alternatives are l, r

%{\bf Gene} & {\bf GeneID} & {\bf Length} \\ 
% & denotes the end of a cell/column, \\ changes to next table row
%\hline % draws a line under the column headers

%human latexin & 1234 & 14.9 kbps \\
%mouse latexin & 2345 & 10.1 kbps \\
%rat latexin   & 3456 & 9.6 kbps \\
% Watch out. Every line must have 3 columns = 2x &. 
% Otherwise you will get an error.

%\end{tabular}
%\caption[title of table]{\textbf{title of table} - Overview of latexin genes.}
% You only need to write the title twice if you don't want it to appear in bold in the list of tables.
%\label{latexin_genes} % label for cross-links with \ref{latexin_genes}
%\end{table}
\begin{table}[Clasificaci\'{o}n de Peque\~{n}os Sat\'{e}lites]
\centering
\begin{tabular}{ | c | c |}
	\hline
	Clasificaci\'{o}n & Masa \\ \hline
	Mini-Sat\'{e}lite & 100-500kg \\
	Micro-Sat\'{e}lite & 10-100kg \\
	Nano-Sat\'{e}lite & 1-10kg \\ 
	Pico-Sat\'{e}lite & 0.1-1kg \\
	Femto-Sat\'{e}lite & $ < $ 100gr \\ \hline
\end{tabular}
\caption[Clasificaci\'{o}n de Peque\~{n}os Sat\'{e}lites]{\textbf{Clasficaci\'{o}n de Peque\~{n}os Sat\'{e}lites}}
\end{table}


\section{Planteamiento del Problema}
El desarrollo e investigaci\'{o}n en sistemas de software y hardware para sat\'{e}lites artificiales dentro de universidades es un tema acad\'{e}mico muy costoso; ya que usualmente se suelen planear proyectos para sat\'{e}lites de gran volumen, con tiempo de fabricaci\'{o}n a m\'{a}s de un a\~{n}o, un muy reducido tiempo de vida en el exterior y caracter\'{i}sticas de funcionamiento que  muchas veces son innecesarias para un proyecto realizado por estudiantes.


Por lo cual, en ocasiones este tipo de proyectos no puede verse funcionando en una \'{o}rbita al exterior del planeta. Principalmente por el costo total de fabricaci\'{o}n, lanzamiento al espacio mediante cohetes de transporte, colocaci\'{o}n en \'{o}rbita y notificaciones ante autoridades coordinadoras (‘Secretar\'{i}a de Comunicaciones y Transportes’, ‘Comisi\'{o}n Federal de Telecomunicaciones’, 'International Maritime Organization',  'International Civil Aviation Organization', 'Uni\'{o}n Internacional de Telecomunicaciones', ‘International Amateur Radio Union’ [4] y “Amateur Satellite Corporation’ [6] ). 


En M\'{e}xico los problemas existentes en las universidades y sector industrial para impulsar la investigaci\'{o}n en exploraci\'{o}n espacial y telecomunicaciones satelitales son: la escasez o inexistencia de presupuesto, tiempo, recursos humanos e intelectuales para estas \'{a}reas de conocimiento.


Otro aspecto importante es la calidad en la ense\~{n}anza y aprendizaje en las universidades p\'{u}blicas que ofrecen carreras de telecomunicaciones, el material de aprendizaje debe ser compartido entre los alumnos de un mismo grupo (antenas, transceptores, cables, conectores, decodificadores).

\section{Descripci\'{o}n de la Propuesta}
Un sistema de comunicaciones por sat\'{e}lite se divide en dos componentes: La estaci\'{o}n terrena y el sat\'{e}lite en \'{o}rbita. [10] La estaci\'{o}n terrena debe contar con al menos un transceptor de radio VHF y UHF, antenas para las bandas de radio frecuencia VHF y UHF, computadora personal para el procesamiento de telemetr\'{i}a y programaci\'{o}n de actividades para el sat\'{e}lite, conectores y adaptadores.


El segundo componente de la propuesta es el pico-sat\'{e}lite con un sistema b\'{a}sico, conocido como M\'{o}dulo de Servicio. Compuesto por los subsistemas de:

\begin{itemize}
\item Estructura
\item Telemetr\'{i}a
\item Alimentaci\'{o}n El\'{e}ctrica
\item Control Termal
\item Control de Posici\'{o}n
\end{itemize}

Y cumplir\'{a} con los requisitos generales para sat\'{e}lites CanSat que previamente han sido marcados.


La propuesta de trabajo consiste en la implementaci\'{o}n de los componentes, una estaci\'{o}n terrena y un pico-sat\'{e}lite CanSat.


Al concluir este trabajo de investigaci\'{o}n, la replicaci\'{o}n de la propuesta, por estudiantes de la Facultad de Telem\'{a}tica servir\'{a} para el aprendizaje y reafirmaci\'{o}n de conocimiento en las \'{a}reas de electr\'{o}nica, desarrollo de software, telecomunicaciones y direcci\'{o}n de proyectos.

\section{Preguntas de Investigaci\'{o}n}
\begin{itemize}
\item ¿Los resultados obtenidos ayudar\'{a}n a futuras investigaciones sobre peque\~{n}os sat\'{e}lites en el pa\'{i}s?
\item ¿El trabajo realizado podr\'{a} ser replicado por otros estudiantes de la Universidad de Colima?
\item ¿Los estudiantes que repliquen el proyecto se interesar\'{a}n por la radio afici\'{o}n?
\item ¿Qu\'{e} tecnolog\'{i}as implementar\'{a} este proyecto de investigaci\'{o}n?
\item ¿Qu\'{e} mejoras a tecnolog\'{i}as se podr\'{a}n generar al concluir de manera exitosa este proyecto de investigaci\'{o}n?
\item ¿Los costos de fabricaci\'{o}n y pruebas del CanSat ser\'{a}n costeables para incluirse como parte de los proyectos integradores en la Ingenier\'{i}a en Telem\'{a}tica?
\end{itemize}

\section{Objetivos}
\subsection{Objetivo General}
Implementar  un pico-sat\'{e}lite con el tama\~{n}o de una lata para bebidas de 12oz. con capacidades de env\'{i}o y recepci\'{o}n de informaci\'{o}n, o CanSat, para ser utilizado en experimentos cient\'{i}ficos, acad\'{e}micos y de prop\'{o}sito general.

\subsection{Objetivos Espec\'{i}ficos}
\begin{itemize}
\item Desarrollar un pico-sat\'{e}lite.
\item Colocar el CanSat en un vuelo sub-orbital a una distancia de al menos 20 mt de altura.
\item Documentar los est\'{a}ndares y procedimientos empleados para la fabricaci\'{o}n de sistemas aeroespaciales.
\item Adquirir informaci\'{o}n de telemetr\'{i}a mediante un enlace de radio a una estaci\'{o}n terrena una vez iniciado el viaje sub-orbital del prototipo.
\item Desarrollar materiales educativos que faciliten el conocimiento sobre exploraci\'{o}n espacial, sistemas de telecomunicaciones de uso amateur y comerciales basados en sat\'{e}lites artificiales, protecci\'{o}n y uso adecuado del entorno natural a partir de la fabricaci\'{o}n de sat\'{e}lites artificiales. 
\end{itemize}

\section{Hip\'{o}tesis}
Los resultados obtenidos con desarrollo del CanSat incrementaran el inter\'{e}s de los estudiantes de la Facultad de Telem\'{a}tica en la Universidad de Colima para continuar con el desarrollo e investigaci\'{o}n de sistemas de comunicaciones aeroespaciales.

\section{Justificaci\'{o}n}
Ante las dificultades dadas a conocer en el planteamiento del problema, la investigaci\'{o}n acad\'{e}mica puede verse beneficiada con la fabricaci\'{o}n y uso de peque\~{n}os sat\'{e}lites (clasificados como Nano, Pico y Femto sat.), siendo posible la investigaci\'{o}n y experimentaci\'{o}n de sistemas espaciales para estudiantes. El resultado del trabajo de investigaci\'{o}n, puede ser colocado a alturas de 1-20 km  mediante globos meteorol\'{o}gicos (globos de Helio) o a alturas de 20-1,000 mt mediante cohetes de modelismo espacial para representar una misi\'{o}n espacial.  Y a su vez, el peque\~{n}o sat\'{e}lite debe interactuar con las condiciones extremas de clima al funcionar fuera del globo terr\'{a}queo. 
Se pueden obtener grandes trabajos con un presupuesto limitado y los retos t\'{e}cnicos de optimizar cada recurso del sat\'{e}lite.

\section{Factibilidad}
Para realizar de manera exitosa \'{e}ste proyecto debe tener el sustento de factibilidad econ\'{o}mica y t\'{e}cnica. La factibilidad econ\'{o}mica determina que no se exceder\'{a} un presupuesto fijo o que se necesite una suma muy grande de dinero para solventar los gastos por el trabajo. La factibilidad t\'{e}cnica determina las capacidades del investigador y los recursos f\'{i}sicos o digitales requeridos.

\subsection{Factibilidad Econ\'{o}mica}
Un proyecto o misi\'{o}n bajo las gu\'{i}as de Competencia Cansat provee la oportunidad de experimentar el dise\~{n}o y ciclo de vida de un sistema aeroespacial. Dichas normas, de manera general, establecen un costo no mayor a de 1000 USD. El prototipo tendr\'{a} un costo menor al l\'{i}mite por ser fabricado con componentes que previamente se ten\'{i}an y otros pueden comprarse con descuentos para estudiante.

\subsection{Factibilidad T\'{e}cnica}


Hoy en d\'{i}a se tiene alcance a herramientas necesarias para el dise\~{n}o y fabricaci\'{o}n de dispositivos electr\'{o}nicos dentro de la Universidad de Colima. El acceso a Internet y el comercio electr\'{o}nico permiten la b\'{u}squeda y compra de componentes o materiales electr\'{o}nicos y electricos que agoten su existencia en las ciudades de Colima o Guadalajara; por lo cual no existe riesgo por una investigaci\'{o}n inconclusa debido a material faltante. El conocimiento de las herramientas de dise\~{n}o y fabricaci\'{o}n ha sido aprendido en cursos dentro de esta facultad.


%: ----------------------- HELP: special characters
% above you can see how special characters are coded; e.g. $\alpha$
% below are the most frequently used codes:
%$\alpha$  $\beta$  $\gamma$  $\delta$

%$^{chars to be superscripted}$  OR $^x$ (for a single character)
%$_{chars to be suberscripted}$  OR $_x$

%>  $>$  greater,  <  $<$  less
%≥  $\ge$  greater than or equal, ≤  $\ge$  lesser than or equal
%~  $\sim$  similar to

%$^{\circ}$C   ° as in degree C
%±  \pm     plus/minus sign

%$\AA$     produces  Å (Angstrom)




% dextran, starch, glycogen continued


%: ----------------------- HELP: references
% References can be links to figures, tables, sections, or references.
% For figures, tables, and text you define the target of the link with \label{XYZ}. Then you call cross-link with the command \ref{XYZ}, as above
% Citations are bound in a very similar way with \cite{XYZ}. You store your references in a BibTex file with a programme like BibDesk.




%: ----------------------- HELP: adding figures with macros
% This template provides a very convenient way to add figures with minimal code.
% \figuremacro{1}{2}{3}{4} calls up a series of commands formating your image.
% 1 = name of the file without extension; PNG, JPEG is ok; GIF doesn't work
% 2 = title of the figure AND the name of the label for cross-linking
% 3 = caption text for the figure

%: ----------------------- HELP: www links
% You can also see above how, www links are placed
% \href{http://www.something.net}{link text}

% variation of the above macro with a width setting
% \figuremacroW{1}{2}{3}{4}
% 1-3 as above
% 4 = size relative to text width which is 1; use this to reduce figures





%: ----------------------- HELP: lists
% This is how you generate lists in LaTeX.
% If you replace {itemize} by {enumerate} you get a numbered list.


 


%: ----------------------- HELP: tables
% Directly coding tables in latex is tiresome. See below.
% I would recommend using a converter macro that allows you to make the table in Excel and convert them into latex code which you can then paste into your doc.
% This is the link: http://www.softpedia.com/get/Office-tools/Other-Office-Tools/Excel2Latex.shtml
% It's a Excel template file containing a macro for the conversion.

%\begin{table}[htdp]
%\centering
%\begin{tabular}{ccc} % ccc means 3 columns, all centered; alternatives are l, r

%{\bf Gene} & {\bf GeneID} & {\bf Length} \\ 
% & denotes the end of a cell/column, \\ changes to next table row
%\hline % draws a line under the column headers

%human latexin & 1234 & 14.9 kbps \\
%mouse latexin & 2345 & 10.1 kbps \\
%rat latexin   & 3456 & 9.6 kbps \\
% Watch out. Every line must have 3 columns = 2x &. 
% Otherwise you will get an error.

%\end{tabular}
%\caption[title of table]{\textbf{title of table} - Overview of latexin genes.}
% You only need to write the title twice if you don't want it to appear in bold in the list of tables.
%\label{latexin_genes} % label for cross-links with \ref{latexin_genes}
%\end{table}



% There you go. You already know the most important things.


% ----------------------------------------------------------------------



